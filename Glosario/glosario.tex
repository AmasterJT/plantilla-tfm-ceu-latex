% =========================
% Glosario de términos
% =========================

\newglossaryentry{backend}{
    name=Backend,
    description={Conjunto de servidores y componentes software encargados de proporcionar servicios, gestionar la lógica de negocio y permitir el acceso a recursos y datos desde aplicaciones cliente}
}

\newglossaryentry{framework}{
    name=Framework,
    description={Estructura de desarrollo compuesta por componentes de software reutilizables, librerías y convenciones que facilitan la creación de aplicaciones de forma estandarizada y eficiente}
}

\newglossaryentry{resultset}{
    name=ResultSet,
    description={Estructura de datos utilizada en JDBC para representar el conjunto de resultados devuelto por una consulta a una base de datos, permitiendo recorrer fila a fila los registros obtenidos y acceder a los valores de cada columna}
}

\newglossaryentry{jdbc}{
    name=JDBC,
    description={Java Database Connectivity, API estándar de Java que permite a las aplicaciones acceder y gestionar bases de datos relacionales mediante la ejecución de consultas SQL y la manipulación de los resultados obtenidos}
}

\newglossaryentry{sku}{
    name=SKU,
    description={Stock Keeping Unit, identificador único de un producto en el inventario}
}


\newglossaryentry{dao}{
    name=DAO,
    description={Data Access Object, patrón de diseño que encapsula el acceso a la base de datos, separar el acceso a la base de datos de la lógica de negocio y proporciona una interfaz para realizar operaciones CRUD sobre una base de datos}
}

\newglossaryentry{fxml}{
    name=FXML,
    description={Es un archivo XML que permite describir el diseño y la estructura visual de una aplicación JavaFX}
}

\newglossaryentry{base de datos relacional}{
    name=Base de datos relacional,
    description={Una base de datos relacional es un conjunto de una o más tablas estructuradas en registros (líneas) y campos (columnas), que se vinculan entre sí por campos en común que poseen las mismas características en ambas tablas, como por ejemplo el nombre de campo, tipo y longitud. A estos campos generalmente se les denomina campos identificadores (ID) o campos clave. A esta forma de construir bases de datos se le denomina modelo relacional.}
}


% =========================
% Glosario de siglas
% =========================

\newacronym
  [plural=WMS,
   longplural={Sistemas de Gestión de Almacenes}]
  {wms}{WMS}{Sistema de Gestión de Almacenes}

\newacronym{api}{API}{Interfaz de Programación de Aplicaciones}
\newacronym{rest}{REST}{Transferencia de Estado Representacional}
\newacronym{erp}{ERP}{Planificación de Recursos Empresariales}
\newacronym{gui}{GUI}{Interfaz Gráfica de Usuario}
\newacronym{sql}{SQL}{Lenguaje de Consulta Estructurado}
\newacronym{http}{HTTP}{Hypertext Transfer Protocol}
\newacronym{https}{HTTPS}{Hypertext Transfer Protocol Secure}
\newacronym{json}{JSON}{JavaScript Object Notation}
\newacronym{xml}{XML}{Extensible Markup Language}

% =========================
% Impresión del glosario
% =========================

\section*{Glosario de siglas}
\label{sec:glosario-siglas}
\vspace{-1.3cm}

\printglossary[type=\acronymtype, title={}, toctitle={}]

\section*{Definiciones}
\label{sec:glosario-terminos}
\vspace{-1.3cm}

\printglossary[title={}, toctitle={}]
